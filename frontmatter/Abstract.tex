%!TEX root = ../Dissertation.tex
\begin{center}
   Numerical Model of Cilia-Driven Transport of Inhaled Particles in the Periciliary Layer of the Human Tracheobronchial Tree
\end{center}\vspace{-1cm}
%https://tex.stackexchange.com/questions/247158/using-hyperref-with-addcontentsline#comment588227_247162
% errors with section numbers not linking correctly with hyperref
\phantomsection 
\addcontentsline{toc}{section}{Abstract}
\section*{Abstract}\label{sec:abstract}

The human respiratory system has evolved to maintain critical functions for survival despite being vulnerable to threats in the environment that are inhaled as part of the necessary gas exchange process. In the respiratory tract, ciliated cells line the tracheal-bronchial tree where mucus producing cells create a barrier that captures inhaled particles. The ciliated cells have hair like protrusions that beat in complex beat patterns in order to move the mucus towards the throat where it is expectorated or swallowed. The system complexity has made it difficult to properly measure particle clearance \invivo. The subject of this dissertation is to numerically study the interaction of inhaled particles with the ciliated cell population. More specifically, the numerical modeling effort builds on work analyzing mucosal layer clearance by addition of arbitrary shaped finite-sized particles in the mucosal layer. In our work, we show that finite-sized particles compared to massless tracers have non-trivial clearance paths, challenging and enriching previous modeling assumptions that massless tracer particles are sufficient for measuring clearance.

We consider one specific simplified model, a single cilium infinitely repeating, and a more representative model, a cilium patch, to understand the effect that biophysical airway parameters have on particle-cilia interaction. We consider the effects that the configuration parameters have on clearance and, in addition, how the dynamics of a finite-sized particle differ from massless tracers. Second, we broaden our model to consider a patch of cilia that represent a two-dimensional cross section of the upper airway. The effect of multiple cilia structures are studied by varying arrangement and synchrony. Next, we quantify in detail the effect that these parameters have on the transport of any arbitrary shaped particles. This dissertation is concluded with a thorough analysis of the flow dynamics in the periciliary layer by incorporating three distinct measures of mixing.
