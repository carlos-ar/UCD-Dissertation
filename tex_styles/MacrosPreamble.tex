% Use this file to load additional packages or define macro
% commands that will be used in the dissertation.
%

\usepackage{graphicx}
\usepackage{caption} %%% optional but probably necessary
\usepackage{subcaption} %%% optional but probably necessary

%% -- CAR added packages
%% All of the packages below are for a specific reason, and I
%% added the comments next to the package

%% -- For generating text to test template
\usepackage{blindtext}

%% -- Since I changed from standard book from amsbook, I needed to get
%% the page layout to look the same, so I used the package layout
%% to print out the margins and then I did some maths to get the
%% margins to fit within UCD's limits
\usepackage{layout}
% -- new lengths
\setlength{\footskip}{18pt}
\setlength{\textheight}{9in}

%% -- Nice looking tables
%% I use https://www.tablesgenerator.com/ when coming up with ideas
\usepackage{booktabs}
%% -- % I preferred to use a different title and section formatting,
%% and to do so I ended up using the titlesec package.
%% The formatting specific commands are below
\usepackage{titlesec}
\usepackage{titletoc}
% -- for colors mainly to highlight pieces
\usepackage{xcolor}
% -- JP asked for a flow chart, and instead of making a ppt or word
%% document, I said "why not learn tikz?", but given that this was
%% 2 weeks before the deadline, I ended up using
%% https://www.mathcha.io/editor
\usepackage{tikz}

%% -- Some commands for works I did not want to re-type
\DeclareMathOperator{\erfc}{ercf}
\newcommand{\invivo}{\emph{in-vivo}}
\newcommand{\invitro}{\emph{in-vitro}}
\newcommand{\flowrate}{$\mu$m\textsuperscript{2}/s}
\newcommand{\velocity}{$\mu$m/s}
\newcommand{\pmten}{PM\textsubscript{10}}
\newcommand{\pmtwofive}{PM\textsubscript{2.5}}
\newcommand{\pmultra}{PM\textsubscript{0.1}}


%% -- 'titlesec' has a lot of fine tuning, so I took the defaults and played around
%% with it a bit.
%% Directly took this from the titlesec documentation

\titleformat{\chapter}[display]
{\normalfont\huge\bfseries}{\chaptertitlename\ \thechapter}{20pt}{\Huge}
\titleformat{\section}
{\normalfont\Large\bfseries\singlespacing}{\thesection}{1em}{}
\titleformat{\subsection}
{\normalfont\large\bfseries}{\thesubsection}{1em}{}
\titleformat{\subsubsection}
{\normalfont\normalsize\bfseries}{\thesubsubsection}{1em}{}
\titleformat{\paragraph}[runin]
{\normalfont\normalsize\itshape}{\theparagraph}{1em}{---}[:~~~]
\titleformat{\subparagraph}[runin]
{\normalfont\normalsize\bfseries}{\thesubparagraph}{1em}{}

%% defaults for titlesec package
\titlespacing*{\chapter} {0pt}{5pt}{4pt}
\titlespacing*{\section} {0pt}{3.5ex plus 1ex minus .2ex}{2.3ex plus .2ex}
\titlespacing*{\subsection} {0pt}{3.25ex plus 1ex minus .2ex}{1.5ex plus .2ex}
\titlespacing*{\subsubsection}{0pt}{3.25ex plus 1ex minus .2ex}{1.5ex plus .2ex}
\titlespacing*{\paragraph} {0pt}{3.25ex plus 1ex minus .2ex}{1em}
\titlespacing*{\subparagraph} {\parindent}{3.25ex plus 1ex minus .2ex}{1em}

%% -- 'tikz' package setup, essentially copied after generating the 
%% flowchart in mathcha.io
\usetikzlibrary{shapes.geometric, arrows}
\tikzstyle{startstop} = [rectangle, rounded corners, minimum width=3cm, minimum height=1cm,text centered, draw=black, fill=red!30,text width = 3cm]
\tikzstyle{io} = [trapezium, trapezium left angle=70, trapezium right angle=110, minimum width=3cm, minimum height=1cm, text centered, draw=black, fill=blue!30]
\tikzstyle{process} = [rectangle, minimum width=3cm, minimum height=1cm, text centered, draw=black, fill=orange!30]
\tikzstyle{decision} = [diamond, minimum width=3cm, minimum height=1cm, text centered, draw=black, fill=green!30]
\tikzstyle{arrow} = [thick,->,>=stealth]
